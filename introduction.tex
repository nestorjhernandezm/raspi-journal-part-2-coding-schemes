%!TEX root = raspi_journal.tex
% At the beginning, there was darkness and then... bang! \ac{NC}
% \cite{ahlswede2000network} appeared to save us from the evilness
% of routing.
%
% General introduction. Introduction to topic addressed in the journal.
% Review of the State of the Art. Specify why our approach has benefits
% and which are they. Indicate contributions.

% RAW INTRODUCTION: NEED OF DISTRIBUTED NETWORKS AND THE RASPBERRY PI
Due to the advent of the \ac{IoT}, approximately fifty million devices
ranging from sensors to phones are expected to be connected through
data networks in relatively short period of time \textbf{[REFERENCE]}.
This massive deployment requires the design and testing of new
distributed systems that permit to cope with the amount of traffic
from the proposed services provided by these devices. Therefore,
development platforms that help to quickly deploy, analyze and
evaluate this type of settings are highly desirable for research.
The emergence of a powerful and inexpensive portable computer such
as the Raspberry Pi, running standard \ac{OS} like Linux or Windows,
allows to achieve this goal since it enables the deployment
of distributed systems at a low cost. Also, the use of an \ac{OS} that
has a large user community helps to manage this networks in less time
\textbf{INCLUDE PRIOR WORK}.

% INTRODUCTION TO NETWORK CODING
Introduced in~\cite{ahlswede2000network}, \ac{NC}
constitutes a paradigm shift in the way networks are understood
by changing how information is sent through them.
Instead of treating the packets as atomic
unmodifiable units, packets are seen as algebraic entities in a \ac{GF}
that can be operated on to create new coded packets. This permits to
remove the limitation of sending \textit{specific} packets by now sending
coded packets as \textit{linear equations} of the original ones. This
change in the way of seeing how the data is represented, brings new
features that can be exploited. In this way, instead of typically encoding
and decoding on a hop basis, relaying nodes can take part in the
coding process without needing to decode. Therefore, a relay can
\textit{recode} packets, i.e. encode again previously received encoded
(but not decoded) packets in order to reduce delay and still
take advantage of the data representation for the next hop.

Compared to other broadly used coding schemes, e.g. \ac{LDPC}
codes~\cite{gallager1962low} or Reed-Solomon
codes~\cite{reed1960polynomial}, network coding is a technology that
has been implemented in real systems since the early years of its
conception. Katti et al.~\cite{katti2008xors} showed the
performance of the COPE protocol for the two-way relay channel
in a wireless network, which relied on minimalistic coding and obtaining
gains over a forwarding scheme. Later, Pedersen et
al.~\cite{pedersen2008implementation} used commercially available
Symbian OS mobile phones to implement network coding in a
\ac{D2D} cooperation-based application. Furthermore, researchers at Aalborg
University developed Kodo~\cite{kodo2011pedersen}, a C++11 network
coding library intended to make network coding functionalities easily
available for both the research community and commercial entities.
Based on Kodo is CATWOMAN~\cite{hundeboll2012catwoman}, a protocol
implemented on top of the BATMAN protocol~\cite{johnson2008simple} for
WiFi multi-hop meshed networks using some of the intuition from
COPE, but deployed within the X topology with overhearing and
currently available as open source in the Linux kernel. Many other
implementations have been tested on real world systems~
\cite{pahlevani2013playncool,katti2008xors,krigslund2013core,paramanathan2013leanandmean}.

In this work, we provide detailed measurements of the goodput and energy
consumption of \ac{Raspi} models 1 and 2, when performing network coding
operations with different codecs based on \ac{RLNC} \cite{ho2006random}
such as: full \ac{RLNC}, multicore-enabled \ac{RLNC}, sparse \ac{RLNC}
and tunable sparse \ac{RLNC}. For this work, we assess the \ac{Raspi}
performance excluding the channel effect in the processing, meaning that
we do not consider packet losses or delays in order to have a
description of the codes in terms of their parameters. To achieve this,
we perform a measurement campaign with the indicated coding schemes and
their parameters in our local Raspberry Pi testbed. Our measurements permit
us to characterize the processing speed (goodput) and energy consumption
of these devices showing that speeds of 800 Mbps and processing energy per bit
values of 0.1 nJ are possible.

Our work is organized as follows. Section~\ref{sec:schemes} defines the
coding schemes employed in our study. Later, in Section~\ref{sec:metrics}
we describe the considered metrics and methodology for performance comparison
of the codes deployed in the \ac{Raspi}. In Section~\ref{sec:measurements}, we
show the measurements in the \ac{Raspi} models of the mentioned metrics
providing full discussions about the observed operational regimes and effects.
Final conclusions and future work are reviewed in
Section~\ref{sec:conclusions}.

%The introduction should briefly place the study in a broad context and highlight why it is important. It should define the purpose of the work and its significance. The current state of the research field should be reviewed carefully and key publications should be cited. Please highlight controversial and diverging hypotheses when necessary. Finally, briefly mention the main aim of the work and highlight the main conclusions. As far as possible, please keep the introduction comprehensible to scientists outside your particular field of research.
