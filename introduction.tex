%!TEX root = raspi_journal.tex
% At the beginning, there was darkness and then... bang! \ac{NC}
% \cite{ahlswede2000network} appeared to save us from the evilness
% of routing.
%
% General introduction. Introduction to topic addressed in the journal.
% Review of the State of the Art. Specify why our approach has benefits
% and which are they. Indicate contributions.

% RAW INTRODUCTION: NEED OF DISTRIBUTED NETWORKS AND THE RASPBERRY PI
Due to the advent of the \ac{IoT}, approximately 50 billion devices
ranging from sensors to phones are expected to be connected through
data networks in relatively short period of time \cite{cisco2011forecast}.
This massive deployment requires the design and testing of new
distributed systems that permit to manage the amount of traffic
from the proposed services provided by these devices. Therefore,
development platforms that help to quickly deploy, analyze and
evaluate this type of scenarios are highly desirable for research.
With the emergence of the \ac{Raspi}, a lrelatively powerful low-cost
computer with the size of a credit card, these evaluations are becoming
possible now. This platform has been used as a general purpose hardware
for \ac{IoT} applications as reported in surveys such as
\cite{alfuqaha2015iot}. In these applications, the
\ac{Raspi} might be the sensoring or computing entity (or even both) for
a required task. To achieve this, it can be extended from being a
simple computer using self-designed or already available extension modules.

A benefit of using the \ac{Raspi} as development platform is its large
community of supporters. By running standard operating systems such as
Linux or Windows, this permits to utilize standard, well tested and
reliable tools to administrate and maintain these networks in
a flexible, stable and supported manner which is a major requirement
to make a scalable deployment. Moreover, by enabling system designers to
configure and deploy several devices at the same time, possible deployments
of tens, hundreds or even thousands of Raspberry Pi's would allow to
analyze representative data patterns of the \ac{IoT}. Different use cases of
\ac{IoT} applications employing the \ac{Raspi} as a building block can be found
in the literature. A basic study of the \ac{Raspi} as a possible device
for sensor applications can be found in \cite{vujovic2014raspberry}.
Authors in \cite{kruger2015rapid} consider using the \ac{Raspi} as an
\ac{6LoWPAN} gateway for a set of sensor and mobile devices to an
\ac{IP} version 6 network. In \cite{mahmoud2016sensorian}, a general
purpose sensoring platform for \ac{IoT} is presented using the \ac{Raspi}
as its basic building block. Various interesting \ac{IoT} use cases
are the works presented in \cite{wirz2015enabling,alletto2016indoor,
jalali2016fog,ueyama2014exploiting}. In \cite{wirz2015enabling}, many
\ac{Raspi}s are used as controllable smart devices
with network connectivity that may ubiquously interact with different users,
each represented by a mobile device through a smartphone application.
The study in \cite{alletto2016indoor} considers \ac{Raspi}s as a data
processing unit for disseminating artwork content in smart museums.
An \ac{IoT} setting where the \ac{Raspi} is employed as a nano-server
in distributed storage and computing can be found in \cite{jalali2016fog}.
Finally, in \cite{ueyama2014exploiting}, the authors present
the \ac{Raspi} as the processing entity of an unmanned aereal
vehicle application to increase the resilience of a wireless sensor networks.
However, even though all these advances in the \ac{IoT} area regarding
the exploitation of the \ac{Raspi} capabilities, the application data
is forwarded using former conventional routing methods which may
not satisfy the need of a distributed network for \ac{IoT}
applications as mentioned earlier 

% INTRODUCTION TO NETWORK CODING
In this context, introduced in~\cite{ahlswede2000network}, \ac{NC}
constitutes a paradigm shift in the way data networks are understood
by changing how information is sent through them and stored at the end devices.
Instead of treating the packets as atomic
unmodifiable units, packets are seen as algebraic entities in a \ac{GF}
that can be operated on to create new coded packets. This permits to
remove the limitation of sending \textit{specific} packets by now sending
coded packets as \textit{linear equations} of the original ones. This
change in the way of seeing how the data is represented, brings new
features that can be exploited. In this way, instead of typically encoding
and decoding on a hop basis, relaying nodes can take part in the
coding process without needing to decode. Therefore, a relay can
\textit{recode} packets, i.e. encode again previously received encoded
(but not decoded) packets in order to reduce delay and still
take advantage of the data representation for the next hop. This new type
of coding across the network is proven to achieve the multicast
capacity \cite{ahlswede2000network,koetter2003algebraic}.

Compared to other broadly used coding schemes, such as \ac{LDPC}
codes~\cite{gallager1962low} or Reed-Solomon
codes~\cite{reed1960polynomial}, network coding is a technology that
has been studied and implemented in real systems since the early years of
its conception. A decentralized network code that has been proven
to achieve the multicast capacity with very high probability is
\ac{RLNC} \cite{ho2006random}. Later, a work observing the
benefits of employing \ac{RLNC} in meshed networks is the MORE protocol
addressed in \cite{chachulski2007more}. Shortly afterwards, authors
in~\cite{katti2008xors} showed the performance of an implementation
of the COPE protocol for the two-way relay channel
in a wireless network, which relied on minimalistic coding and obtaining
gains over a forwarding scheme. Later, the work
in~\cite{pedersen2008implementation} used commercially available
Symbian OS mobile phones to implement network coding in a
\ac{D2D} cooperation-based application. Furthermore,
in~\cite{kodo2011pedersen}, the Kodo library was introduced. Kodo is
a C++11 network coding library intended to make network coding
basic functionalities easily available for both the research community
and commercial entities. Based on Kodo, the CATWOMAN
protocol~\cite{hundeboll2012catwoman} is implemented on top of the
BATMAN protocol~\cite{johnson2008simple} for WiFi multi-hop meshed
networks. It uses some of the intuition from COPE, but it is deployed
within the X topology with overhearing links. Its source code is available
as open source in the Linux kernel. Moreover, many other successful
implementations have been tested on real world systems such as
found in~\cite{pahlevani2013playncool,krigslund2013core,
paramanathan2013leanandmean,seferoglu2011i2nc}. For the Raspberry Pi device, an
evaluation of \ac{RLNC} can be found in \cite{paramanathan2014sharing}.
However, this evaluation focused particularly on observing the achievable
speeds only for \ac{RLNC} with different configurations of
the code parameters. In this previous work, a relevant practical aspect that
was not evaluated was the use of hardware acceleration through \ac{SIMD} or
the multi-core capabilities of more advanced \ac{Raspi} models. These features are
becoming more frequent in new processors to largely increase their computational
power for the upcoming demand. In this sense, the Raspberry Pi posses an \ac{ARM}
architecture that could be multi-core as mentioned and also exploits the \ac{SIMD}
feature with the optimized NEON instruction set, but still to the best of our knowledge
there has been no documentation in the literature about these capabilities.

Therefore, in this work we provide detailed measurements of the goodput (processing
speed) and energy consumption of \ac{Raspi} models 1 and 2, when
performing network coding operations with different codecs based on \ac{RLNC}
such as: full dense \ac{RLNC}, multi-core enabled \ac{RLNC}, sparse \ac{RLNC}
and tunable sparse \ac{RLNC}. For these coding schemes, the encoder and decoder
implementations from Kodo are able to detect and make use of the \ac{SIMD}
through the NEON instruction set of the Raspberry Pi by recognizing the \ac{ARM}
architecture with its multicore capabilities. We assess
the \ac{Raspi} performance excluding the effect of packet losses or delays in
order to have a description of the processing and energy consumption for only the
codes in terms of their parameters. To achieve this, we perform a measurement
campaign with the indicated coding schemes and their parameters in various models
of a Raspberry Pi device. Our measurements permit us to characterize the mentioned
metrics of these devices showing that processing peeds of 800 Mbps and processing
energy per bit values of 0.1 nJ are possible. Our work is organized as follows.
Section~\ref{sec:schemes} defines the
coding schemes employed in our study. Later, in Section~\ref{sec:metrics}
we describe the considered metrics and methodology for performance comparison
of the codes deployed in the \ac{Raspi}. In Section~\ref{sec:measurements}, we
show the measurements in the \ac{Raspi} models of the mentioned metrics
providing full discussions about the observed operational regimes and effects.
Final conclusions and future work are reviewed in
Section~\ref{sec:conclusions}.

%The introduction should briefly place the study in a broad context and highlight why it is important. It should define the purpose of the work and its significance. The current state of the research field should be reviewed carefully and key publications should be cited. Please highlight controversial and diverging hypotheses when necessary. Finally, briefly mention the main aim of the work and highlight the main conclusions. As far as possible, please keep the introduction comprehensible to scientists outside your particular field of research.
