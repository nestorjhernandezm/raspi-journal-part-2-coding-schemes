%!TEX root = raspi_journal.tex
% At the beginning, there was darkness and then... bang! \ac{NC}
% \cite{ahlswede2000network} appeared to save us from the evilness
% of routing.
%
% General introduction. Introduction to topic addressed in the journal.
% Review of the State of the Art. Specify why our approach has benefits
% and which are they. Indicate contributions.

% INTRODUCTION TO NETWORK CODING
Network coding is a novel and disruptive technology introduced in the year
2000~\cite{ahlswede2000network}. It has imposed a change of paradigm in the way
how researchers and industry alike interpret networks and, more specifically,
the role of intermediate relays in the process of transmission of information.
The key element is that it allows relays to be part of the coding process
enabling the recoding (i.e., to produce linear combinations without a previous
decoding process) of the received packets. The recoding process disrupts with
the traditional routing (store and forward) strategies in today's networks,
including the Internet \todot{a citation here?}. Network coding allows the
increase of throughput, reliability, security and delay performance of the
networks. In its initial conception, the relays would combine and transfer
packets of different flows, while in later applications of the technology, both
the sources and the relays would form linear combinations of packets within the
same information flow.

Differently from other broadly used coding schemes (e.g., Gallager's
codes~\cite{gallager1962low}, Reed-Solomon codes~\cite{reed1960polynomial}),
network coding is a technology that has been implemented in real systems since
the early years of its conception. Only six years later, in 2006, Katabi et
al.~\cite{katabi2006practical} published the results of the performance of their
protocol COPE when implemented in a real wireless system\todot{is there any
previous implementation?}. Similarly, in 2008, Pedersen et
al.~\cite{pedersen2008implementation} used commercially available Symbian OS
enabled mobile phones to implement network coding in a \ac{D2D} cooperation
application. In 2011, researchers at Aalborg University in Denmark developed
KODO~\cite{kodo2011pedersen}, a C++ network coding library with open source code
for researchers intended to make network coding implementations easily available
for both, the research community and commercial entities.
CATWOMAN~\cite{hundeboll2012catwoman} is a protocol implemented in 2012 on IEEE
802.11 based multi-hop meshed networks. It uses network coding to improve the
throughput of meshed networks established using the
B.A.T.M.A.N.~\cite{johnson2008simple} routing protocol. It is currently
available as open source code and part of the Linux kernel. Also in 2012,
Pahlevani et al. implemented PlayNCool~\cite{pahlevani2013playncool} as an
opportunistic protocol that increases the throughput of a wireless mesh network
by using network coding and local optimization. Similarly, many other
implementations have been tested on real world systems~
\cite{katti2008xors,krigslund2013core,paramanathan2013leanandmean}.

It is then undeniable that network coding theory has always been hand to hand
with practical implementations. However, each one of the previously tested
protocols has been implemented in a separate, somehow hard to reproduce,
testbed. Given the previous importance of practical testbeds for this particular
technology, the needs for reproducibility of past and future results among
researchers, and the needs of low-cost and easy-to-deploy testbeds, this paper
aims to solve those needs. In this work, the authors contribute to the research
community by providing detailed instructions on how to set up a low-cost,
heterogeneous and scalable testbed consisting of Raspberry Pis (Raspis), and by
providing detailed measurements of goodput and energy consumption of such
testbed when performing network coding operations with different codecs (such as
full RLNC, multicore-enabled RLNC, sparse RLNC and tunable sparse RLNC). The
goals are \textbf{i)} to provide information on how to easily build and deploy a
low-cost testbed for researchers and industry alike in order to test and
implement their network coding related protocols; and \textbf{ii)} to provide
information on the expected performance in terms of energy and goodput of such
testbed while covering a wide range of codecs.

The heterogeneity of the testbed comes from the fact that old, single core,
Raspis (i.e., Raspberry Pi 1 model B) are integrated seamlessly in the described
testbed together with newer, multicore versions (i.e., Raspberry Pi 2 model B).
The testbed is scalable because tenths or hundreds of Raspis can be connected
together with the described method. And it is low-cost because Raspis represent
a low fraction of the cost of laptops, desktop computers or other similar
devices. \todot{is all of this true"?}

Our work is organized as follows: Section~\ref{sec:testbed} introduces the
testbed system, its setup, connectivity, files configuration and compilation of
the Kodo library. Section~\ref{sec:schemes} defines the coding schemes employed
in our study. Later, in Section~\ref{sec:metrics} we describe the considered
metrics for performance comparison of the codes deployed in the \ac{Raspi}. In
Section~\ref{sec:measurements}, we show the measurements in the \ac{Raspi} of
the mentioned metrics. Final conclusions and future work are reviewed in
Section~\ref{sec:conclusions}.


%The introduction should briefly place the study in a broad context and highlight why it is important. It should define the purpose of the work and its significance. The current state of the research field should be reviewed carefully and key publications should be cited. Please highlight controversial and diverging hypotheses when necessary. Finally, briefly mention the main aim of the work and highlight the main conclusions. As far as possible, please keep the introduction comprehensible to scientists outside your particular field of research.
