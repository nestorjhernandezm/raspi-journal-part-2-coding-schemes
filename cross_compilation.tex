
%\subsection{Kodo cross-compilation: From your PC to the Raspberry Pi}

WRITE INTRO OF WHY WE WANT TO CROSS COMPILE. E.G. 1) WE DO NOT HAVE MONITORS
CONNECTED TO THE RASPIS, 2) A PC IS SIGNIFICANTLY FASTER AND FAR BETTER EQUIPPED (RAM, CPU, HARDDRIVE), 3) MORE CONVINIENT TO DO EVERYTHING IN ONE PC AND DISTRIBUTE IT TO THE RASPIS AFTERWARDS. MORE?

\subsection{Install toolchain on PC}

Besides the previous description (\textbf{Include compiling Kodo from the
RasPi from the scratch}), the testbed administrator can compile Kodo in its
personal workstation and transfer the generated binaries directly to
a path in the \ac{Raspi}. To achieve this, we get a toolchain that
contains the binaries for the \texttt{raspberry-gxx49-arm-g++} compiler
for the \ac{Raspi}. Therefore, we strongly recommend any testbed
administrator to do the following procedure. In what follows, we provide
the instructions considering that the NFS server uses the \texttt{\$HOME}
directory as the working directory. However, the administrator may choose
some other working directory of its preference if desired.

\begin{enumerate}

\item Create toolchain directory:
\begin{lstlisting}[]
$ TOOLCHAINDIR="${HOME}/toolchains"
$ mkdir -p $TOOLCHAINDIR
$ cd $TOOLCHAINDIR
\end{lstlisting}
\FloatBarrier
\vspace{-5mm}

\item Download the \ac{Raspi} toolchain for 64-bit Linux from.
A more recent \ac{Raspi} toolchain may be available at
\texttt{http://buildbot.steinwurf.dk/toolchains/linux/}:

\begin{lstlisting}[]
$ TOOLCHAIN="raspberry-gxx49-arm"
$ wget http://kom.aau.dk/project/TuneSCode/raspi/${TOOLCHAIN}.zip
\end{lstlisting}
\FloatBarrier
\vspace{-5mm}

\item Extract the downloaded file:
%locally in the NFS server. After
%this operation, there should be a new directory for the toolchain
%in the server. \\
\begin{lstlisting}[]
$ unzip ${TOOLCHAIN}.zip
\end{lstlisting}
\FloatBarrier
\vspace{-5mm}

\item (Optional) Delete zip file:
%locally in the NFS server. After
%this operation, there should be a new directory for the toolchain
%in the server. \\
\begin{lstlisting}[]
$ rm ${TOOLCHAIN}.zip
\end{lstlisting}
\FloatBarrier
\vspace{-5mm}

\item We can now verify that the ARM cross compiler is working:

\begin{lstlisting}[]
$ ${TOOLCHAINDIR}/${TOOLCHAIN}/bin/${TOOLCHAIN}-g++ --version
raspberry-gxx49-arm-g++ (crosstool-NG 1.21.0) 4.9.3 20150311 (prerelease)
Copyright (C) 2014 Free Software Foundation, Inc.
This is free software; see the source for copying conditions.  There is NO
warranty; not even for MERCHANTABILITY or FITNESS FOR A PARTICULAR PURPOSE.
\end{lstlisting}
\FloatBarrier
\vspace{-5mm}

%\begin{lstlisting}[]
%$ echo "${TOOLCHAINDIR}/${TOOLCHAIN}/bin/${TOOLCHAIN}-g++"
%/home/<USER>/toolchains/raspberry-gxx49-arm/bin/raspberry-gxx49-arm-g++
%\end{lstlisting}
%\FloatBarrier
%\vspace{-5mm}
    
%    The ARM cross compilers should now be located in

\item Make toolchain binaries available systemwide. Instead of calling
the ARM cross compiler using its full path, we can make it accessible
from the command shell systemwide. One way to do this is by adding the 
toolchain binaries directory to the Linux environment variable \texttt{PATH}
when the \ac{OS} starts up:

%\item Add toolchain binaries to \texttt{PATH}. Instead specifying the full
%path to the toolchain binaries we can instead tell the operation system
%where to search for it. This makes the toolchain binaries available
%systemwide.

\begin{lstlisting}[]
$ echo PATH=\"\$PATH:${TOOLCHAINDIR}/${TOOLCHAIN}/bin\" >> ${HOME}/.profile
\end{lstlisting}
\FloatBarrier
\vspace{-5mm}
%$ printf 'PATH="%s/%s/bin"\n' "${TOOLCHAINDIR}" "${TOOLCHAIN}" >> ${HOME}/.profile

%\item Add the \texttt{bin} folder of the toolchain to the \texttt{PATH}
%Linux environment variable of the server. This will help the server OS
%to recognize the location of the compiler command, which will be needed
%later. To do so, edit the \texttt{\$HOME/.profile} to add in a newline:
%\texttt{PATH="\$PATH:\$HOME/raspberry-gxx49-arm/bin"}. Save the
%\texttt{\$HOME/.profile}. \\

\texttt{.profile} should now contain the line we inserted. There may be more
code in you file.
% MAC and Hostname file
\Suppressnumber\begin{lstlisting}[]
<@\textcolor{gray}{\$HOME/.profile}@>
<@\textcolor{gray}{
---------------------------------------------------------------}
\Reactivatenumber @>
PATH="$PATH:/home/<USERNAME>/toolchains/raspberry-gxx49-arm/bin"
\end{lstlisting}
\FloatBarrier

\item Update \texttt{PATH}. Source \texttt{.profile} to make the changes
take effect in your system:
\begin{lstlisting}[]
$ source ${HOME}/.profile
\end{lstlisting}
\FloatBarrier
\vspace{-5mm}

\item ARM cross compiler should now be available systemwide:

\begin{lstlisting}[]
$ raspberry-gxx49-arm-g++ --version
\end{lstlisting}
\FloatBarrier
\vspace{-5mm}

%\item Restart the server session in order for the changes made in the
%previous step take effect. To verify this, open a new terminal and type:
%\texttt{raspberry-gxx49-arm-g++ --version}. A correct binary installation
%should return an output similar to:
%
%\texttt{raspberry-gxx49-arm-g++ (crosstool-NG 1.21.0) 4.9.3 20150311 (prerelease)
%Copyright (C) 2014 Free Software Foundation, Inc.
%This is free software; see the source for copying conditions.  There is NO
%warranty; not even for MERCHANTABILITY or FITNESS FOR A PARTICULAR PURPOSE.} \\

\end{enumerate}

\subsection{Cross compile example}

Create a directory to hold our code:
\begin{lstlisting}[]
$ CODEDIR="${HOME}/code"
$ mkdir -p ${CODEDIR}
$ cd ${CODEDIR}
\end{lstlisting}
\FloatBarrier
\vspace{-5mm}

Create a \texttt{hello\_world.cpp} using your favorite text editor:

\Suppressnumber\begin{lstlisting}[]
<@\textcolor{gray}{\${CODEDIR}/hello\_world.cpp}@>
<@\textcolor{gray}{
---------------------------------------------------------------}
\Reactivatenumber @>
#include <iostream>

int main()
{
    std::cout << "Hello World!" << std::endl;
    return 0;
}
\end{lstlisting}
\FloatBarrier

Save \texttt{hello\_world.cpp} and compile it for \ac{Raspi}:

\begin{lstlisting}[]
$ raspberry-gxx49-arm-g++ hello_world.cpp -o hello_world
\end{lstlisting}
\FloatBarrier
\vspace{-5mm}

Lets copy the executable file to a \ac{Raspi} and test it. We will use \ac{SCP}
to copy the executable to one of the \ac{Raspi}s.

Default username is "pi" and password is "raspberry"

\begin{lstlisting}[]
$ scp main pi:<RASP_IP>:~/
\end{lstlisting}
\FloatBarrier
\vspace{-5mm}

If you do not know the \ac{IP} address of a \ac{Raspi} in your network, you can
connect a monitor to it and run the following command after you have logged in:
\begin{lstlisting}[]
pi@rasp01:~ $ ifconfig
eth0      Link encap:Ethernet  HWaddr b8:27:eb:72:77:54  
          inet addr:192.168.87.107  Bcast:192.168.87.255  Mask:255.255.255.0
          inet6 addr: fe80::e0a5:38f3:6f82:bc79/64 Scope:Link
          UP BROADCAST RUNNING MULTICAST  MTU:1500  Metric:1
          RX packets:1537 errors:0 dropped:0 overruns:0 frame:0
          TX packets:445 errors:0 dropped:0 overruns:0 carrier:0
          collisions:0 txqueuelen:1000 
          RX bytes:259117 (253.0 KiB)  TX bytes:52551 (51.3 KiB)

\end{lstlisting}
\FloatBarrier
\vspace{-5mm}

After the executable has been copied to a \ac{Raspi}. Then, \ac{SSH} to it:

\begin{lstlisting}[]
$ ssh pi:<RASP_IP>
\end{lstlisting}
\FloatBarrier
\vspace{-5mm}

We can list the directory content after we have logged into the \ac{Raspi} and
see that \texttt{hello\_world} is there:

\begin{lstlisting}[]
pi@rasp07:~ $ ls
hello_world  rasp_config
\end{lstlisting}
\FloatBarrier
\vspace{-5mm}

Now, simply execute the \texttt{hello\_world}:

\begin{lstlisting}[]
pi@rasp07:~ $ ./hello_world
Hello World!
\end{lstlisting}
\FloatBarrier
\vspace{-5mm}

\subsection{Cross compile Kodo}

WRITE SOME SMALL INTRO AND MENTION HOW TO REQUEST A KODO LICENSE

\begin{enumerate}

\item Create a directory to hold our code:
\begin{lstlisting}[]
$ CODEDIR="${HOME}/code"
$ mkdir -p ${CODEDIR}
$ cd ${CODEDIR}
\end{lstlisting}
\FloatBarrier
\vspace{-5mm}

\item Clone the Kodo repository and change directory into the repository: \\
\begin{lstlisting}[]
$ git clone https://github.com/steinwurf/kodo-cpp.git
$ cd kodo-cpp
\end{lstlisting}
\FloatBarrier
\vspace{-5mm}

\item Configure Kodo to build executables for the \ac{ARM} architecture using our toolchain:
\begin{lstlisting}[]
$ python waf configure --cxx_mkspec=cxx_raspberry_gxx49_arm
...
'configure' finished successfully (0.620s)
\end{lstlisting}
\FloatBarrier
\vspace{-5mm}

\item Build executables:
\begin{lstlisting}[]
$ python waf build
...
'build' finished successfully (2m22.918s)
\end{lstlisting}
\FloatBarrier
\vspace{-5mm}

\item SHARED and/or STATIC libraries
\item COPY TO RASPI
\item RUN EXECUTABLE encode\_decode\_simple

%\item Navigate to the repository and configure \texttt{waf} by typing:
%\texttt{python config.py} and select the 16th ``make specification'' file
%for the \ac{Raspi}, e.g. option \texttt{[16]cxx\_raspberry\_gxx49\_arm}
%presented by the file.
%
%This command configures \texttt{waf} to use the proper compiler and its
%required flags to generate the binaries for the \ac{Raspi}. If the
%configuration was correct, the output will indicate:
%\texttt{'configure' finished successfully (X.XXXs)}, where \texttt{X.XXX}
%is total time in seconds for configuring the project in the server. \\
%
%\item Execute \texttt{python waf build}. If the build process was
%successful, the generated binaries for the \ac{Raspi} should be located
%in \texttt{build/cxx\_raspberry\_gxx49\_arm} in the Kodo repository.
%\textbf{Indicate how the binary files should look like}.
%
%Once this procedure is made, the testbed administrator can relocate the
%generated binary files to the \ac{Raspi}s through the network as desired
%by using the \texttt{scp} command during the configuration step.


\end{enumerate}

\subsection{Kodo Builds for the \ac{Raspi}, Platform Support and Documentation}

You can check the build status of Kodo, Fifi and other relevant projects
through their respective repository master branch on our buildbot page
\cite{steinwurf2016buildbot}. Our buildbot displays the status of the builds
for Raspbian 8 and GCC 4.9 for the ARM architecture which is the relevant one
for the \ac{Raspi}. At the link, you can check build status and build
statistics. Also, documentation about Kodo basics with a tutorial is available
at \cite{kododocs}.






\subsection{Kodo python}

WRITE HOW TO RUN A KODO PYTHON SCRIPT IN THE RASPIS

\subsection{fabric}

SHORT INTRO. EXPLAIN A FEW USECASES FOR FABRIC. E.G. REBOOT, INSTALL PACKAGE, RUN SCRIPT ON MULTIPLE RASPIS. COPY FILES TO/FROM PC AND MULTIPLE RASPIS 


% Exit chroot, umount, and write to memory card
\begin{lstlisting}[]
from fabric.api import run, env, task
# Python Fabric script to run commands on multiple hosts through ssh
#
# Run script as 'fab <task>', where <task> is one of the scripts functions
# marked as a tesk. The task marked as 'default' will be run if <task> is not
# specified

env.hosts = ['rasp00.domain.com','rasp01.domain.com','rasp02.domain.com']
env.user = 'root'
env.password = 'sdn'

@task
def install(program):
    """
    Install a program
    program: program name
    """
    result = run('apt-get install -y {}'.format(program), quiet=True)

@task
def copy_to_rasp(filename):
    put(...)

\end{lstlisting}
\FloatBarrier

execute a script function by calling "fab install:feh"




\subsection{Long running jobs over SSH (PUT THIS SECTION SOMEWHERE ELSE)}

Problem when you logout, then applications will terminate. SCREEN is the answer.

EXAMPLE

\texttt{hello\_world} returns immediately. This is not always the case.
Sometimes it is desired to run simulations that should run for days. 
A software package called \texttt{screen} can be used for this purpose.
Simply run ...
