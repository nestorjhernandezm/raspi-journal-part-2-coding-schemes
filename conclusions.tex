\label{sec:conclusions}
Given the need for a \ac{Raspi} testbed easy to configure and maintain
for network coding applications where results could be reproducible
in a simple fashion, we provide a testbed description at Aalborg
University with an in-depth performance evaluation for network coding
schemes to indicate the processing capabilities and energy behaviour
for two \ac{Raspi} models both of using their \ac{SIMD} capabilities
and considering a multicore implementation for the \ac{Raspi} 2. Our
findings report that processing speeds of more than X Mbps and Y Mbps are attainable for \textbf{[goodput scenarios]}, respectively, and that
processing energy per bit for network coding is below 1 nJ in a large regime
and even an order of magnitude lower for some algorithms with the same power consumption for \textbf{[goodput scenarios]}. Future work in this area,
will be to evaluate further elaborated codecs, expand the setup and
automation of tasks within the testbed creation, consider other working
topologies or packet erasures and include a description for network
connectivity where the testbed has more autonomy with \ac{IP} address assignment.