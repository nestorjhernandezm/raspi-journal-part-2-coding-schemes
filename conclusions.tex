\label{sec:conclusions}
Given the usefulness of the \ac{Raspi} as a low-complex
processing node in large-scale networks and network coding
techniques against state-of-the-art routing, we provide
a performance evaluation of network coding schemes focusing on
processing speed and energy consumption for two \ac{Raspi} models.
The evaluation includes algorithms that exploit both \ac{SIMD}
instructions and multicore capabilities of the \ac{Raspi} 2. Our
measurements show that processing speeds of more than 80 Mbps and 800
Mbps are attainable for the \ac{Raspi} model 1 and 2, respectively,
for a wide range of network coding configurations and maintaining
a processing energy below 1 nJ/bit (or even an order of magnitude lower)
in similar configurations. For the use of multithreading, we quantify
processing gains ranging from 2$\times$ for $g = 16$ to 13$\times$ for
$g = 1024$ when employing 4 threads each in a different core. Future
work in the use of \ac{Raspi} devices will focus on considering: (i)
the performance of the \ac{Raspi} in scenarios with synthetic packet
losses, (ii) wireless networks where real packet losses can occur
and (iii) other topologies such as broadcast or
cooperative scenario to compare with theoretical results in order to
evaluate the performance of different network codeswith the \ac{Raspi}.